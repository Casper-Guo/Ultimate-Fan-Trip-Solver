\documentclass[11pt]{article}

% Mathematical packages
\usepackage{amsmath}
\usepackage{amssymb}
\usepackage{mathtools}

% Formatting packages
\usepackage[margin=1in]{geometry}
\usepackage{enumerate}
\usepackage{hyperref}

\title{Integer Linear Programming Formulation}
\date{}

\begin{document}

\maketitle

\section{Definitions}
Let $E$ be the set of events and $T$ be the set of teams one wishes to see. Denote $n_e = |E|$ and $n_t = |T|$. Number the events in chronological order from $1$ to $n_e$ and additionally define a dummy event $e_0$.

Define $\mathbf{C}_{n_e \times n_e}$ to be the cost matrix according to some measure where $C_{ij}$ is the cost of travelling from $e_i$ to $e_j$. Note that this matrix is upper triangular in the sense that the elements below the main diagonal are infinite/undefined since one cannot travel back in time. Additionally define $C_{0j} = 0 \; \forall j$ and $C_{i0} = 0 \; \forall i$ to allow for no-cost travel to and from the dummy event.

Define $\mathbf{M}_{n_e \times n_t}$ to be the matchup matrix where $M_{ij} = 1$ if team $t_j$ participates in event $i$ and 0 otherwise. Note that this matrix is very sparse (with density no more than $2/n_t$) since each row, representing a single event, will involve exactly two teams and have either one or two entries equal to 1 depending on the formulation. To solve the "watch one team play some other teams" problem, only the other team should receive a non-zero entry. To solve the "watch games played by a set of teams" problem, both teams in an event should receive a non-zero entry. Additionally define $M_{0j} = 0 \; \forall j$ to represent the dummy event.

Note that if, instead of optimizing for a set of teams one wishes to see, we instead optimize for a set of venues one wishes to visit, we can define a venue matrix $\mathbf{V}$ exactly analogous to $\mathbf{M}$ and all subsequent derivations apply similarly.

\section{Variables}
Define variables $x_{ij} \in \{0, 1\}$ for all $0 \leq i \leq n_e$ and $0 \leq j \leq n_e$ if and only if it is feasible to travel directly from event $e_i$ to event $e_j$ (i.e., $j > i$ or $j = 0$). The variable $x_{ij}$ is 1 if one travels directly from event $e_i$ to event $e_j$ and 0 otherwise. The subset of the variables between non-dummy events can also be represented as a matrix $\mathbf{X}_{n_e \times n_e}$.

Define the visit order variables $u_i \in \mathbb{R}$ where $0 \leq i \leq n_e$. $u_i = 0$ if event $e_i$ is not visited and $u_i = k$ if event $e_i$ is the $k$-th event visited in the trip.

\section{Objective Function}
Minimize:

\begin{equation} \label{eq:objective}
\sum_{i=1}^{n_e} \sum_{j=i+1}^{n_e} (\mathbf{C}_{ij} + 1) x_{ij}
\end{equation}

The meaning of the $+1$ is to minimize the number of events visited. This is useful in scenarios where the cost is otherwise 0, such as the driving distance between two MLB games in the same series.

\section{Constraints}
Constraints for the range of variables:

\begin{equation} \label{constr:x}
    x_{ij} \in \{0, 1\} \; \forall 0 \leq i \leq n_e, 1 \leq j \leq n_e, j > i
\end{equation}

\begin{equation} \label{constr:x_dummy}
    x_{i0} \in \{0, 1\} \; \forall 1 \leq i \leq n_e
\end{equation}

Constraints to make sure we get a tour:

\begin{equation} \label{constr:in_edge}
    \sum_{i=0, i < j}^{n_e} x_{ij} \leq 1 \; \forall 0 \leq j \leq n_e
\end{equation}

\begin{equation} \label{constr:out_edge}
    x_{i0} + \sum_{j=1, j > i}^{n_e} x_{ij} \leq 1 \; \forall 0 \leq i \leq n_e
\end{equation}

Since we are not guaranteed to visit all events, we need to explicitly restrict the indegree and the outdegree of each node to be equal.

\begin{equation} \label{constr:equal_degree}
    \sum_{i=0, i < j}^{n_e} x_{ij} - \sum_{k=1, k > j}^{n_e} x_{jk} - x_{j0} = 0 \; \forall 0 \leq j \leq n_e
\end{equation}

Finally, to make sure we see all teams at least once:

\begin{equation} \label{constr:teams}
    \sum_{i=0}^{n_e} \sum_{j=i + 1}^{n_e} x_{ij}M_{jk} \geq 1 \; \forall 1 \leq k \leq n_t
\end{equation}

\section{Notes}
We have dropped the subtour elimination constraint of the \href{https://en.wikipedia.org/wiki/Travelling_salesman_problem#Miller%E2%80%93Tucker%E2%80%93Zemlin_formulation}{MTZ formulation} because of the special structure of our graph. The induced subgraph of non-dummy events is directed with no back edges and acyclic. Therefore any cycle on the graph must go through the dummy event and subtours are not possible.

\section{Conclusion}
We have constructed an integer program that is a modified version of the MTZ formulation for TSP. This program has on the order of $n_e^2 / 2$ binary variables and roughly $3n_e + n_t$ constraints.

\end{document}