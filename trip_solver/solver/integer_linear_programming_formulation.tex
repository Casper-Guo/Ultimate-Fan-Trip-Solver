\documentclass[11pt]{article}

% Mathematical packages
\usepackage{amsmath}
\usepackage{amssymb}
\usepackage{mathtools}

% Formatting packages
\usepackage[margin=1in]{geometry}
\usepackage{enumerate}
\usepackage{hyperref}

\title{Integer Linear Programming Formulation}
\date{}

\begin{document}

\maketitle

\section{Definitions}
Let $E$ be the set of events and $T$ be the set of teams one wishes to see. Denote $n_e = |E|$ and $n_t = |T|$. Number the events in chronological order from $1$ to $n_e$ and additionally define a dummy event $e_0$.

Define $\mathbf{C}_{n_e \times n_e}$ to be the cost matrix according to some measure where $C_{ij}$ is the cost of travelling from $e_i$ to $e_j$. Note that this matrix is upper triangular in the sense that the elements below the main diagonal are infinite/undefined since one cannot travel back in time. Additionally define $C_{0j} = 0 \; \forall j$ and $C_{i0} = 0 \; \forall i$ to allow for no-cost travel to and from the dummy event.

Define $\mathbf{M}_{n_e \times n_t}$ to be the matchup matrix where $M_{ij} = 1$ if team $t_j$ participates in event $i$ and 0 otherwise. Note that this matrix is very sparse (with density equal to $1/n_t$) since each row, representing a single event, will involve exactly one team and thus have exactly one entry equal to 1. This matrix is so defined under the assumption that a fan requires that his own team to be playing in the game, thus only distinguishing the opposing team is necessary. Additionally define $M_{0j} = 0 \; \forall j$ to represent the dummy event.

\section{Variables}
Define variables $x_{ij} \in \{0, 1\}$ for all $0 \leq i \leq n_e$ and $0 \leq j \leq n_e$ if and only if it is feasible to travel directly from event $e_i$ to event $e_j$ (i.e., $j > i$ or $j = 0$). The variable $x_{ij}$ is 1 if one travels directly from event $e_i$ to event $e_j$ and 0 otherwise. The subset of the variables between non-dummy events can also be represented as a matrix $\mathbf{X}_{n_e \times n_e}$.

Define the visit order variables $u_i \in \mathbb{R}$ where $0 \leq i \leq n_e$. $u_i = 0$ if event $e_i$ is not visited and $u_i = k$ if event $e_i$ is the $k$-th event visited in the trip.

\section{Objective Function}
Minimize:

\begin{equation} \label{eq:objective}
\sum_{i=1}^{n_e} \sum_{j=i+1}^{n_e} \mathbf{C}_{ij} x_{ij}
\end{equation}

\section{Constraints}
Constraints for the range of variables:

\begin{equation} \label{constr:x}
    x_{ij} \in \{0, 1\} \; \forall 0 \leq i \leq n_e, 1 \leq j \leq n_e, j > i
\end{equation}

\begin{equation} \label{constr:x_dummy}
    x_{i0} \in \{0, 1\} \; \forall 1 \leq i \leq n_e
\end{equation}

The visit order variables can be modelled as continuous variables to improve solver performance, although in practice they will be almost equal to integers once optimized.

\begin{equation} \label{constr:u}
    1 \leq u_i \leq n_t + 1 \in \mathbb{R} \; \forall 0 \leq i \leq n_e
\end{equation}

Constraints to make sure we get a tour:

\begin{equation} \label{constr:in_edge}
    \sum_{i=0, i < j}^{n_e} x_{ij} \leq 1 \; \forall 1 \leq j \leq n_e
\end{equation}

\begin{equation} \label{constr:out_edge}
    x_{i0} + \sum_{j=1, j > i}^{n_e} x_{ij} \leq 1 \; \forall 1 \leq i \leq n_e
\end{equation}

The subtour elimination constraint:

\begin{equation} \label{constr:order}
    u_i - u_j \leq (1 - x_{ij})(n_t + 1) - x_{ij} \; 0 \leq i \neq j \leq n_e
\end{equation}

To make sure we start and end the tour at the dummy event $e_0$:

\begin{equation} \label{constr:start}
    u_0 = 1
\end{equation}

\begin{equation} \label{constr:start_out_edge}
    \sum_{j=1}^{n_e} x_{0j} = 1
\end{equation}

\begin{equation} \label{constr:end}
    \sum_{i=1}^{n_e} x_{i0} = 1
\end{equation}

Finally, to make sure we see all teams once and exactly once:

\begin{equation} \label{constr:teams}
    \sum_{i=0}^{n_e} \sum_{j=i + 1}^{n_e} x_{ij}M_{jk}  = 1 \; \forall 1 \leq k \leq n_t
\end{equation}

In other words, all columns of $\mathbf{X} \mathbf{M}$ sum to 1.

\section{Proofs}
\subsection{No Subtour}
Since we do not care to visit all events, the tour order variable's definition is different from that seen in the \href{https://en.wikipedia.org/wiki/Travelling_salesman_problem#Miller%E2%80%93Tucker%E2%80%93Zemlin_formulation}{MTZ formulation}.

Compare the following MTZ constraint to our constraint \eqref{constr:order}:

\begin{equation} \label{constr:mtz_u}
    1 \leq u_i \leq n_e \quad 1 \leq i \leq n_e
\end{equation}

\begin{equation} \label{constr:mtz}
    u_i - u_j + 1 \leq (n_e - 1)(1 - x_{ij}) \; 1 \leq i \neq j \leq n_e
\end{equation}

When $x_{ij} = 0$, the MTZ constraint simplifies to $u_i - u_j \leq n_e - 1$ which is just a reformulation of the bounds placed on $u$. In our constraint this simplifies to $u_i - u_j \leq n_t + 1$ which has the same meaning, albeit with a different bound to allow for a tour of $n_t$ events with some events left unvisited.

When $x_{ij} = 1$, the MTZ constraint simplifies to $u_i - u_j \leq -1$, which guarantees that $e_i$ is visited before $e_j$ as expected. In our constraint this simplifies to $u_i - u_j \leq -1$ which is equivalent.

\subsection{Tour Ordering}
We want for any unvisited event $e_i$ to have $u_i = 0$. This constraint is actually implied by constraint \eqref{constr:teams}, which can be rewritten as.

\begin{equation}
    \sum_{i=0}^{n_e} \sum_{j = 1, j > i}^{n_e} \sum_{k=1}^{n_t} x_{ij} M_{jk} = n_t
\end{equation}

Consider any fixed $i, j$ where $j \neq 0$, there is exactly one $k$ such that $M_{jk} = 1$. Thus constraint \eqref{constr:teams} implies that there is exactly $n_t$ non-zero $x_{ij}$ variables where $j \geq 1$, corresponding to the $n_t$ non-dummy events visited. Then constraint \eqref{constr:end} provides that there is exactly 1 non-zero $x_{i0}$. So in total we have exactly $n_t + 1$ non-zero $x_{ij}$ variables.

Each non-zero $x_{ij}$ variable correspond to some value of $u_i$ and we therefore have at least $n_t + 1$ non-zero $u_i$ variables. Constraint \eqref{constr:order} provides that the absolute difference between any two of these variables is at least 1 and constraint \eqref{constr:u} provides that they range between 1 and $n_t + 1$. Thus these $n_t + 1$ variables must be exactly the integers between 1 and $n_t + 1$ and there can be no other non-zero $u_i$ variables.

\section{Conclusion}
We have constructed an integer program that is a modified version of the MTZ formulation for TSP. This program has roughly $n_e^2 / 2 + n_e$ binary variables and roughly $n_e^2 + 3n_e + n_t$ constraints.

\end{document}